\chapter{Testing Plan}

The purpose of this system is to provide a secure, private, efficient, reliable, and constantly accessible data backup system.  Therefore, any failures could be critical, possibly causing people or corporations to lose their data and resulting in millions of dollars in damages.  It is our responsibility to ensure that each individual component of the system works perfectly, and also that the system as a whole smoothly functions.

Since our intended system is very complex and consists of several modules, we will most likely follow an incremental process model, and use the continuous integration approach.  Under this method, performing unit testing is very intuitive, as each team member will continuously test their own modules in isolation as they develop them.  We will remember to test early and often in order to catch bugs as soon as possible.  Part of this will be white box testing for code coverage; at the very least, we should make sure all  linearly independent paths through the code are tested.  Furthermore, we will make careful documentation of all bugs we catch.  This practice will help us get a better idea of common failure points in our modules, so we can focus more on those areas when we test later on.  At least once per month, we will peer-review each others' modules in order to ensure good coding practices, including high cohesion and low coupling.  Once we have a working system, we will perform black box testing with team members using the system for their own personal backups, and sharing files amongst the four of us.  Finally, we will simulate large numbers of PCs sharing files by using virtual machines.

Besides testing within the team, we should also open testing up to focus groups representing hypothetical users.  Once our system reaches a fully-working version, we will find friends and family members to try using our system.  The obvious problem is that we cannot ask them to rely fully on our system as a replacement to the backup systems they already use, because if our system fails that will represent a great inconvenience and loss to them.  Therefore, we will ask them to use our system in addition to whatever backups they currently use.  Our final system relies on a large network of users sharing files with each other using a peer-to-peer protocol; therefore at some point we should open the system up to the general public for beta testing.  Once again, we must put a disclaimer as clearly as possible that this system cannot yet guarantee privacy and reliability.  A public beta testing would definitely require the most effort to implement and maintain, but it will be necessary before we can release any type of product.  We may not reach this phase in the course of this year, but if we continue the project later we definitely will.