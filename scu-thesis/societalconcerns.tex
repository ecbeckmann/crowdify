\chapter{Societal Issues}

\section{Ethical Discussion, Social Concerns, and Compassion}

\subsection{Justification}
	There are many ethical justifications for our project.  In the larger sense, we hope to make the world a better place by providing a better data backup and restore service than what is currently out there.  The bottom line is that our product will provide a convenience to the general public and assist them with storing files and also protecting against the problems of data loss.  Specifically we will help corporations become more efficient in their operations, which could bring economic prosperity and higher standards of living for society as a whole.  This is all in agreement with the first rule of the ACM Code of Ethics to ''contribute to society and human well-being.'' \cite{acmethics}

	Our project also deals directly with the human right to privacy.  As detailed in rule 1.7 of the ACM code of ethics, we are to ''respect the privacy of others.'' \cite{acmethics}  Furthermore, privacy is listed as a fundamental human right in Article 12 of the Universal Declaration in Human Rights: ''no one shall be subjected to arbitrary interference with his privacy.'' \cite{udhr}  The right to privacy is an inalienable human dignity that we hope to protect.  Privacy is so important because it gives people greater freedom to make their own moral decisions, independent from the judgement of others and shielded from the pressure to conform to their culture.  As one adage goes: ''The time you spend alone with yourself is the most precious time you have.  This is your proving ground.  It’s where you decide who you are, what values you uphold, and ultimately how you are seen in the eyes of yourself and others.'' \cite{wolf}

	The right to privacy is especially pertinent in this day and age, where more and more of everyone’s personal data is being put out online, vulnerable to access by malicious parties.  We hope our service will allow people to protect their privacy, since instead of storing their data in a single corporation’s database, which presupposes full trust in that corporation, and also provides a single point of attack for governments or hackers who wish to seize the data, we will be distributing the encrypted data securely to many locations in separate fragments.  These are not merely hypothetical concerns.  Data stores run by single entities have suffered data breaches many times in the past.  For example, Apple iCloud was hacked in August of 2014, which resulted in many celebrity photographs being leaked. \cite{independent}  Furthermore, the NSA has controversially forced companies like Google and Yahoo to turn over customer data in moves that have widely been called unconstitutional. \cite{gizmodo}  Our system will not be vulnerable to privacy violations like this, because there will no longer be a single party attackers can go after to view people's data.

\subsection{Lifelong Learning}

	Throughout this project, it is also very important that we are mindful of our own growth in moral character and technical skill.  Our society is increasingly dependent on computers, so as computer engineers we have a large role for improving people’s daily lives.  As we go through this project, we will be sure to adhere to the Software Engineering code of ethics.  We will make every effort to put the public interests and the common good first.  We will strive for the highest quality work in all aspects.  We will strive to cultivate our own character, skills, and abilities, and will seek to grow as much as possible through this project.  Finally, we will treat each other with respect during this project.

\subsection{Possible Pitfalls}
	As with most engineering projects, there are certain moral pitfalls that we might encounter along the way.  First of all, we have to consider that potential users of our system are trusting us to keep their data as private as possible.  If we take any shortcuts or make any oversights while designing our system, we could leave security holes that will compromise privacy.  By not keeping people's data private, we expose them to malicious parties such as identity thieves, hackers, fraud artists, and so on.  We would be indirectly responsible for any damage caused, because we made people believe they had privacy which they did not.  And this would break article 1.7 of the ACM Code of Ethics to ''Avoid harm to others''. \cite{acmethics}  It is therefore our duty to ensure privacy in our system to the maximum extent possible.

	There is also the possible pitfall of our system being used by criminals, due to the enhanced privacy it provides.  By providing a system by which users can store their data with complete privacy, we could allow criminals to hide evidence of their activities, obstructing the activities of legitimate law enforcement agencies. The peer-to-peer architecture we use is similar to protocols such as BitTorrent which are far and wide used only for illicit purposes.  Even though it is not our intention that our system would be used for illegal activities, we must consider this possibility.

\section{Political Concerns}
	This is not a public project, so we can mostly disregard political concerns.

\section{Economic considerations and Manufacturability}
	The scope of the project at this stage is not to deploy a large scale product for the public to use, but rather to make a system for the purposes of testing new algorithms and methods of data distribution.  Therefore, we will not consider the costs of deploying a final product at this time.

	As for the costs of prototyping itself, we expect it to be quite low.  The software development costs little to no resources.  We will use a number of Raspberry Pi boards to implement our system.  At the cost of \$40 per board, we expect the total cost of the project to be no more than \$300.  We will support this project with our own personal funds.

\section{Health and Safety}
	This project does not require the use of any heavy machinery, extraordinarily large power sources, etc.  Therefore, we don't need any special safety considerations for this project.  However, we will be using electronic equipment such as PCs and Raspberry Pis in the course of this project, so we should exercise common sense as we would handling any electronic appliance.

\section{Sustainability and Environmental Impact}
	The environmental impact of creating our starting system should be minimal, since it is quite small scale and only uses a few electronic devices.  For the reasons mentioned above we will not yet consider the costs of deployment for a large scale system.

\section{Usability}
	One of the requirements and goals for our system is to make our interface as user-friendly as possible.  We will do this by only providing our user with a basic set of controls, hiding the complexity of the data distribution algorithm from them.  It is also important that our system be easy to set up as well.  We can achieve this by making a good installer as well as utilizing auto-configuration to the fullest extent.  Finally, we will do our utmost to prevent hogging of system resources by our program, so that it runs as transparently in the background as possible.
